\section{Conclusioni}
In questo paper abbiamo visto come l'analisi dei dati sia un processo fondamentale per l'ottimizzazione di qualsiasi attività. Attraverso l'analisi di un dataset relativo al fatturato di un'azienda abbiamo mostrato come sia possibile utilizzare gli strumenti più efficaci per la comprensione dei dati.

In particolare, abbiamo utilizzato R come strumento di analisi dei dati e JavaScript, Hadoop e R per l'implementazione di algoritmi di mapreduce. Grazie all'utilizzo di R, abbiamo creato grafici chiari ed efficaci per la comprensione dei dati, come grafici lineari, box plot e grafici a barre con grafico a torta.

Abbiamo anche illustrato il processo di data science, partendo dall'analisi del problema e del dataset, passando per l'elaborazione di algoritmi di mapreduce fino alla produzione di grafici per rispondere alle necessità del problema iniziale.

Inoltre, abbiamo visto come lo strumento di JavaScript per l'implementazione di algoritmi di mapreduce sia un ottimo strumento didattico per approcciarsi al paradigma di programmazione di mapreduce, soprattutto per chi è alle prime armi e non conosce R o Python. Tuttavia, l'utilizzo di Hadoop come software open source è risultato alquanto complicato da configurare correttamente, soprattutto se si vuole andare ad aggiungere nodi al cluster.

In conclusione, abbiamo apprezzato la possibilità di scalare la propria potenza di calcolo e parallelizzare le informazioni, abbiamo visto come la scelta del linguaggio e delle strategie di implementazione di un algoritmo possano incidere sulle performance e di conseguenza sui costi computazionali di un processo.

La gestione di grandi quantità di dati è una disciplina cruciale in un mondo in cui i dati sono sempre più presenti e utilizzati. Gli strumenti e le metodologie utilizzati in questo documento rappresentano i principi base per l'approccio a questa materia, evidenziando come sia necessario adattarsi alle diverse problematiche proposte.