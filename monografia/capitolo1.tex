\section{Introduzione}
L'analisi del ciclo di vita (LCA) è una tecnica utilizzata per valutare l'impatto ambientale di un prodotto, processo o attività per l'intero ciclo di vita, dall'estrazione delle materie prime alla sua eliminazione. È uno strumento completo che considera tutte le fasi del ciclo di vita di un prodotto, compresa la produzione, l'uso e lo smaltimento.

È particolarmente rilevante dal punto di vista della sostenibilità, poiché permette di valutare l'impatto ambientale di un prodotto o servizio lungo tutto il suo ciclo di vita, evitando che i miglioramenti locali comportino solo lo spostamento dell'impatto ambientale altrove. In questo modo, si può affrontare la causa del problema alla radice e non solo i sintomi.

Con l'aumento della popolazione mondiale e dello sviluppo economico, l'impatto ambientale delle nostre attività sta diventando sempre più evidente.

Questo strumento può essere utilizzato da aziende, organizzazioni e dal settore pubblico per prendere decisioni informate sulla gestione ambientale dei loro prodotti e servizi: emissioni di gas a effetto serra, l'inquinamento dell'aria, dell'acqua e del suolo, e l'uso di risorse naturali. L'analisi del ciclo di vita può anche essere utilizzata per identificare i punti critici del ciclo di vita di un prodotto o servizio, in modo da poterli migliorare.
\subsection{Che cos'è LCA}
Le questioni ambientali stanno giocando un ruolo sempre più importante nel processo decisionale a ogni livello: politico, economico, industriale e individuale. Più di una semplice tendenza passeggera, l'attenzione crescente dedicata ai problemi ambientali deriva da una semplice osservazione: a causa della sua limitata capacità di assorbire gli effetti delle attività umane, l'ambiente stabilisce un limite allo sviluppo della società. Questo limite è già stato raggiunto in molte regioni del pianeta\cite{jolliet2015environmental}.

I concetti di sostenibilità sono costantemente discussi nei titoli, ma diventa sempre più difficile agire. Per garantire un futuro sostenibile, le dichiarazioni e gli studi devono essere seguiti da azioni significative che riducano efficacemente l'impatto ambientale e che possano persino migliorare la situazione. Perché un'azione sia efficiente, devono essere soddisfatte tre condizioni:
\begin{itemize}
    \item Le soluzioni tecnologiche devono essere disponibili.
    \item Diverse soluzioni devono essere priorizzate e selezionate le migliori pratiche, tenendo conto dell'efficienza ambientale, dei costi e dei vincoli economici risultanti.
    \item Le azioni dovrebbero essere ottimizzate per ridurre ulteriormente gli impatti.
\end{itemize}

L'analisi del ciclo di vita (LCA) è uno strumento decisionale che affronta specificamente questa esigenza di selezionare e ottimizzare le soluzioni tecnologiche disponibili. Ciò è fondamentale quando le risorse finanziarie sono limitate, poiché Barlow ha affermato in modo leggermente provocatorio: "Il problema non è come affrontare il singolo problema: l'ingegneria è disponibile o può essere sviluppata per affrontare quello. Piuttosto, il problema è come decidere le priorità. Il mondo semplicemente non può permettersi di fare tutto". L'analisi del ciclo di vita è un complemento agli sviluppi tecnologici, poiché evidenzia quali processi dovrebbero essere migliorati in ordine di priorità.

Rilevante inoltre dal punto di vista della sostenibilità, poiché copre l'intero ciclo di vita di un prodotto o servizio, evitando che i miglioramenti locali comportino solo lo spostamento dell'impatto ambientale altrove. Differisce poi da altri metodi ambientali collegando le prestazioni ambientali alla funzionalità, quantificando le emissioni inquinanti e l'uso di materie prime in base alla funzione del prodotto o del sistema.

Nonostante l'analisi del ciclo di vita presenti molti vantaggi, non è priva di limiti. Analogamente alla contabilità economica per la stima del costo effettivo di un prodotto, la contabilità ecologica richiede un certo numero di ipotesi che devono essere logiche e coerenti. Alcune applicazioni dell'LCA sono state duramente criticate, suggerendo che un determinato metodo di analisi del ciclo di vita è stato selezionato per ottenere i risultati attesi dal finanziatore dello studio.

Per coprire un insieme più ampio di impatti ambientali, è emersa la necessità di contabilizzare le emissioni inquinanti nell'aria, nell'acqua e nel suolo. Ciò ha portato a sviluppi metodologici, inizialmente nell'industria del packaging, che alla fine sono stati applicati a tutti i settori economici, poiché è emerso che il prodotto spesso aveva un impatto molto più grande della sua confezione.
\subsection{Da dove nasce}
Tre organizzazioni sono state coinvolte nello sviluppo e nella standardizzazione dell'analisi del ciclo di vita: la Società di tossicologia ambientale e chimica (SETAC), il Programma delle Nazioni Unite per l'ambiente (UNEP) e la ISO. A partire dai primi anni '90, SETAC ha offerto una piattaforma di scambio scientifico per lo sviluppo dell'analisi del ciclo di vita, ancora in corso oggi attraverso presentazioni di conferenze e gruppi di lavoro.

Dal 2002, l'iniziativa del ciclo di vita è stata un importante quadro istituzionale per lo sviluppo dei metodi dell'analisi del ciclo di vita e il loro utilizzo nell'industria. Lanciata da SETAC e UNEP, questa iniziativa mira a sviluppare e diffondere strumenti pratici per valutare soluzioni, rischi, vantaggi e svantaggi associati a prodotti e servizi durante tutto il loro ciclo di vita. La prima fase si è svolta nel periodo 2002-2007, sviluppando un consenso sulle modalità di approccio del ciclo di vita. Ciò è stato seguito da una seconda fase dal 2007 al 2011, con l'obiettivo di diffondere la consapevolezza e l'utilizzo degli approcci del ciclo di vita in tutto il mondo. La terza fase (dal 2012 ad oggi) sta sviluppando un consenso sugli indicatori di impatto e fornendo orientamenti per l'analisi del ciclo di vita organizzativa, che considera gli impatti del ciclo di vita di una determinata azienda o organizzazione, compresa la fornitura, l'uso e lo smaltimento dei suoi prodotti e servizi.

Durante gli anni '80 e '90, ISO\footnote{ISO produce standard internazionali per la maggior parte dei settori tecnologici. Gli standard ISO sono adattati alle applicazioni industriali e derivano da un consenso tra esperti di vari background, tra cui industria, tecnologia, economia e accademia.} ha pubblicato oltre 350 standard relativi alle questioni ambientali, in particolare, la serie ISO 14000 sui sistemi di gestione ambientale, aggiornando e fornendo un quadro per le aziende per gestire l'impatto ambientale delle loro attività e misurare le loro prestazioni ambientali. La serie ISO 14040 (14040-14049) è dedicata all'analisi del ciclo di vita (Tabella 1.2). Il primo standard (ISO 14040) stabilisce le linee guida per l'esecuzione di un'analisi del ciclo di vita. ISO 14044 ha sostituito ISO 14041, 14042 e 14043 nel 2006 per descrivere le fasi di inventario, valutazione dell'impatto e interpretazione. Esempi della sua applicazione sono presentati in ISO 14047 e 14049, e ISO 14048 descrive il formato di documentazione dei dati.

\subsection{Utilizzo e obbiettivi}
Molte associazioni di imprese e aziende industriali già utilizzano l'approccio LCA nel quadro della sostenibilità. Questo metodo è sempre più utilizzato dall'industria per aiutare a ridurre il carico ambientale complessivo lungo l'intero ciclo di vita dei beni e dei servizi, migliorando la competitività dei prodotti dell'azienda e nella comunicazione con gli organi governativi.

Oltre a supportare la presa di decisioni, l'LCA viene ampiamente utilizzato per migliorare la progettazione del prodotto, ad esempio nella scelta dei materiali, nella selezione delle tecnologie, nei criteri di progettazione specifici e nella considerazione del riciclaggio.

Un'altra possibilità interessante fornita dall'LCA è il benchmarking delle opzioni del sistema di prodotto, che può essere utilizzato anche nella decisione di acquisto e investimenti tecnologici, nei sistemi di innovazione, ecc. È uno strumento unico in quanto fornisce una panoramica dei compromessi a monte e a valle associati alle pressioni ambientali, alla salute umana e al consumo di risorse. Queste informazioni a macro scala completano altre valutazioni sociali, economiche ed ambientali.

Anche il settore pubblico fa uso del pensiero del ciclo di vita nelle consultazioni con gli stakeholder e nell'attuazione delle politiche. Ciò garantisce che si tenga in considerazione il quadro generale nelle valutazioni ambientali orientate alle politiche, considerando i compromessi a monte e a valle. L'LCA contribuisce infatti a una politica di prodotto efficiente fornendo informazioni preziose aggiuntive sulle prestazioni ambientali di beni e servizi.

Diversi sono i livelli di contribuzione nelle strategie politiche come:

\begin{itemize}
    \item supporto all'attuazione della strategia tematica dell'UE sull'uso sostenibile delle risorse naturali\cite{EuropaSustaineble};
    \item strategia tematica sulla prevenzione e il riciclaggio dei rifiuti
    \item definizione di criteri di eco-design, contribuendo alla definizione di obiettivi di prestazione nell'ambito del Piano d'azione per la tecnologia ambientale (ETAP) e per i prodotti ad uso energetico nell'ambito della Direttiva EuP, negli appalti pubblici verdi (GPP) e nelle dichiarazioni ambientali di prodotto (EPD).
\end{itemize}
E' infine importante tenere presente che l'utilizzo dell'analisi del ciclo di vita è un semplice strumento di supporto alla decisione, piuttosto che uno strumento decisionale, poiché ha uno specifico focus. In particolare, tende a escludere gli impatti economici e sociali, nonché la considerazione di questioni ambientali più locali. Pertanto, è necessario utilizzarlo in combinazione con altri strumenti per identificare le aree di miglioramento potenziale.