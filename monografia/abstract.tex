\section*{Abstract}

Il presente paper ha l'obiettivo di esaminare le potenzialità e le performance degli strumenti utilizzati per la gestione e l'analisi dei Big Data. Viene approfondito il processo che porta alla produzione di report grafici, tabelle e schemi a partire dai dati di un problema o di una esigenza. La relazione fornisce informazioni dettagliate su come impostare un progetto, dalla scelta e impostazione dell'ambiente di sviluppo alla contestualizzazione delle decisioni prese. Inoltre, vengono allegati i codici per la generazione dei risultati e dei grafici.

Il progetto in esame mira a soddisfare l'esigenza dell'azienda (fittizzia) STECCAPARAPETUTTI S.R.L. di risolvere un problema di contabilità mediante l'estrazione di alcune feature dai dati. A tal fine, sono stati impiegati strumenti avanzati per l'elaborazione dei dati, tra cui algoritmi di MapReduce e tecniche di visualizzazione avanzata.

L'analisi dei dati è stata eseguita utilizzando il linguaggio R per una prima analisi, seguito dall'utilizzo di un tool scritto in JavaScript per prototipare il sistema di MapReduce e risolvere il problema.